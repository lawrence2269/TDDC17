\documentclass[a4paper,10pt]{article}
\usepackage[utf8]{inputenc}
\usepackage{graphicx} 

\begin{document}

\begin{titlepage}
	\centering
	\includegraphics[width=.6\textwidth]{liu-logo.png}\par
	\vfill
	{\scshape\Large TDDC17 ARTIFICIAL INTELLIGENCE\par}
	{\huge\bfseries Lab 2: Search\par}
	\vspace{1cm}
	{\large\itshape Robin Andersson (roban591) \\ Lawrence Thanakumar Rajappa (lawra776)\par}
	\vfill
	{\large \today\par}
\end{titlepage}

\section{Questions and answers}

\subsection{In the vacuum cleaner domain in part 1, what were the states and actions? What is the branching factor?}

The states are represented by GridPos which in turn is represented by the x and y position of a tile in the world.
This GridPos can then either represent a clean or a dirty tile.
The actions that can be performed are: suck dirt and move east, west, south and north.
The actions give that the branching factor is five.

\subsection{What is the difference between Breadth First Search and Uniform Cost Search in a domain where the cost of each action is 1?}

Since the cost of each action is 1, then the prioritization of the action with the lowest cost of Uniform-Cost Search will
give be same as the nearest action that Breadth First Search gives.
This means that there is no difference between the two.

\subsection{ Suppose that $h1$ and $h2$ are admissible heuristics (used in for example A*). Which of the following are also admissible?
\\a) $(h1+h2)/2$
\\b) $2h1$
\\c) $\max(h1,h2)$}

a) This is just the mean of $h1$ and $h2$ which means that it will never exceed $\max(h1, h2)$.
This is obviously admissible since $h1$ and $h2$ are both admissible, meaning that the mean must be less than the true cost.
\\
b) This is not admissible since $2h1$ can exceed $\max(h1, h2)$ which means that it also can exceed the true cost.
\\
c) As explained in a) this is admissible since it cannot exceed $h1$ or $h2$ and since both of those are admissible then
this heuristics must also be less than the true cost.

\subsection{If one would use A* to search for a path to one specific square in the vacuum domain, what could the heuristic $(h)$ be? The cost function $(g)$? Is it an admissible heuristic?}

A heuristic could be the straight-line distance, since it cannot be overestimated. 
The cost function could be the one for each action that must be performed to reach the goal tile from the initial tile.
Thus the heuristic can never exceed the cost function since the straigh-line distance is the shortest path between to tiles, meaning that the heuristic is admissible.

\subsection{Draw and explain. 
Choose your three favorite search algorithms and apply them to any problem domain (it might be a good idea to use a domain where you can identify a good heuristic function). 
Draw the search tree for them, and explain how they proceed in the searching. 
Also include the memory usage. 
You can attach a hand-made drawing.}

\end{document}