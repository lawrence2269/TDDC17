\documentclass[a4paper,10pt]{article}
\usepackage[utf8]{inputenc}
\usepackage{graphicx} 

\begin{document}

\begin{titlepage}
	\centering
	\includegraphics[width=.6\textwidth]{liu-logo.png}\par
	\vfill
	{\scshape\Large TDDC17 ARTIFICIAL INTELLIGENCE\par}
	{\huge\bfseries Lab 3: Bayesian Networks\par}
	\vspace{1cm}
	{\large\itshape Robin Andersson (roban591) \\ Lawrence Thanakumar Rajappa (lawra776)\par}
	\vfill
	{\large \today\par}
\end{titlepage}

\section{Task 2}

\subsection{Questions and Answers}

\subsubsection{What is the risk of melt-down in the power plant during a day if no observations have been made? What if there is icy weather?}

The risk of melt-down if no observations have been made is 0.02578.
If there is icy weather then the risk is 0.03472.

\subsubsection{Suppose that both warning sensors indicate failure. 
What is the risk of a meltdown in that case? 
Compare this result with the risk of a melt-down when there is an actual pump failure and water leak. 
What is the difference? 
The answers must be expressed as conditional probabilities of the observed variables, P(Meltdown|...).}

The risk of a meltdown when both warning sensors indicate failure is P(Meltdown|PumpFailureWarning, WaterLeakWarning) = 0.14535.
The risk of a meltdown when the failures actually occur is P(Meltdown| PumpFailure, WaterLeak) = 0.20000.
The difference is P(Meltdown| PumpFailure, WaterLeak) - P(Meltdown|PumpFailureWarning, WaterLeakWarning) = 0.05465.

\subsubsection{The conditional probabilities for the stochastic variables are often estimated by repeated experiments or observations. 
Why is it sometimes very difficult to get accurate numbers for these? 
What conditional probabilites in the model of the plant do you think are difficult or impossible to estimate? }

\subsubsection{Assume that the "IcyWeather" variable is changed to a more accurate "Temperature" variable instead (don't change your model). 
What are the different alternatives for the domain of this variable? 
What will happen with the probability distribution of P(WaterLeak | Temperature) in each alternative?}

\section{Task 3}

\subsection{Questions and Answers}

\subsubsection{During the lunch break, the owner tries to show off for his employees by demonstrating the many features of his car stereo.
 To everyone's disappointment, it doesn't work. 
 How did the owner's chances of surviving the day change after this observation?}

 If we do not know if the radio is working then the probability of the owner surviving is P(Survives) = 0.99001.
 If the radio is not working then the probability of the owner surviving is P(Survives) = 0.98116.
 The owner's chance of surviving decreased by 0.00885.

 \subsubsection{How does the bicycle change the owner's chances of survival?}

 The chance that the owner survives when it owns a bike as well is P(survives) = 0.99505.
 The owner's chance of surviving increased by 0.00504 compared to when he did not own the bike.

 \subsubsection{It is possible to model any function in propositional logic with Bayesian Networks. 
 What does this fact say about the complexity of exact inference in Bayesian Networks? 
 What alternatives are there to exact inference? }

 

\end{document}