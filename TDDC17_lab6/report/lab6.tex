\documentclass[a4paper,10pt]{article}
\usepackage[a4paper, total={7in, 8in}]{geometry}
\setlength\parindent{0pt}
\usepackage[utf8]{inputenc}
\usepackage{graphicx} 
\usepackage{amsmath}
\usepackage{amsfonts}
\usepackage{amssymb}
\begin{document}

\begin{titlepage}
	\centering
	\includegraphics[width=.6\textwidth]{liu-logo.png}\par
	\vfill
	{\scshape\Large TDDC17 ARTIFICIAL INTELLIGENCE\par}
	{\huge\bfseries Lab 6: Deep Learning\par}
	\vspace{1cm}
	{\large\itshape Robin Andersson (roban591) \\ Lawrence Thanakumar Rajappa (lawra776)\par}
	\vfill
	{\large \today\par}
\end{titlepage}

\section*{Part 3}

\textbf{Q2. Show the math for why the first Dense layer has 100,480 parameters with these inputs and number of neurons}

The input for the first layer is (28,28) and number of neurons in the dense(hidden) layer is 128 neurons,
Number of biases for hidden layer is 128 neurons, hence
\begin{center}
		(Number of inputs * Number of Neurons in the hidden layer) + Number of biases for hidden layer \\
		(28*28*128)+128 = 100,480 parameters.
\end{center}

\textbf{Q3. Here you will evaluate different mini-bach sizes for stochastic gradient descent (see the deep learning lecture). 
Please separately run the training code above with batch sizes of 1, 10, 100, 1000 and 60000. 
Write down the training times (you can use the first number in seconds, not the per sample time) and 
the training set accuracy reached, both in the first line of the output. 
This can randomly vary a bit between runs but it should give you an idea. 
In your lab report, plot both curves and reason about which batch size produced the most accuracy 
given the time spent, i.e. which batch size would be best to start the training with?}


\begin{center}
	\setlength{\arrayrulewidth}{1.0pt}
	\begin{tabular}{|c|c|c|c|}
		\hline
		 \textbf{Batch Size} & \textbf{Time Duration in (seconds)} & \textbf{Accuracy \%}\\ [1.5ex]
		\hline
		1 & 109 &81.23\\
		\hline
		10 & 15 &82.56\\
		\hline
		100 & 3 &81.78\\
		\hline
		1000 & 2 &72.81\\
		\hline
		60000 & 2 &6.16\\
		\hline
	\end{tabular}
\end{center}


\end{document}